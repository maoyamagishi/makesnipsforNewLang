\documentclass{ltjsarticle}
\usepackage{luatexja} % ltjclasses, ltjsclasses を使うときはこの行不要
\usepackage{amsmath,amssymb}
\usepackage{listings, xcolor}

%%%%%%%%%%%%%%  本文  %%%%%%%%%%%%%%%%%%
\begin{document} 
\section{動機}
VSCodeの拡張機能である、"HyperSnips"をインストールし、「動的スニペットは便利だなぁ」と感激していた。
しかし、各言語ごとの動的スニペットが.hsnipsという独自の拡張子になっており、このファイルを編集するための
コード補完に不満があった。(例えば、globalとendglobalがセットで出てこないなど)\par
各言語ごと、拡張子ごとのスニペットはmanage>>snippetsを開いて選択するようになっており、独自言語やサポート
されていない拡張子に対しては"言語名".jsonを上手く作成できない。\par
今後もVSCodeがサポートしていない拡張子を持つファイルを快適に編集したいと思うことは想像に難くない。
今回、上手くいく方法を見つけたので書き残しておきたい。

\section{方法}
HyperSnipsはJavaScriptと正規表現を用いて動的スニペットを可能にする拡張機能だ。\par
そのため、フォルダの構成を見てみるとsrcや.vscodeなど見慣れたフォルダと同じ階層に
package.jsonというファイルがある。これは自作のシンタックスハイライトやスニペットを
作るときに必要なファイルで、language-configration.json、.tmLanguage.jsonといった言語の
文法が記されたファイルへのパスと一緒にsnipet.jsonファイルのパスを書く。また、ファイパスの
場所にsnipet.jsonファイルを作成しておく。\par
正しく設定されていると、.hsnipsファイルのエディタからmanage>>snippetsでスニペットが選択できるようになる。



%https://qiita.com/OrukRed/items/03f0e38e0f8553ee35d2

\lstset{
    basicstyle = {\ttfamily}, % 基本的なフォントスタイル
    frame = {tbrl}, % 枠線の枠線。t: top, b: bottom, r: right, l: left
    breaklines = true, % 長い行の改行
    numbers = left, % 行番号の表示。left, right, none
    showspaces = false, % スペースの表示
    showstringspaces = false, % 文字列中のスペースの表示
    showtabs = false, % タブの表示
    keywordstyle = \color{blue}, % キーワードのスタイル。intやwhileなど
    commentstyle = {\color[HTML]{1AB91A}}, % コメントのスタイル
    identifierstyle = \color{black}, % 識別子のスタイル 関数名や変数名
    stringstyle = \color{brown}, % 文字列のスタイル
    captionpos = t % キャプションの位置 t: 上、b: 下
}
\begin{lstlisting}[caption=package.jsonの一部]
  "contributes": {
    "languages": [{
        "id": "tyrano",
        "aliases": ["tyranoscript", "tyrano"],
        "extensions": [".ks"],
        "configuration": "./language-configuration.json"
    }],
    "grammars": [{
        "language": "tyrano",
        "scopeName": "source.ks",
        "path": "./syntaxes/tyrano.tmLanguage.json"
    }],
"snippets": [
  {
    "language": "tyrano",
    "scopeName":"source.ks",
    "path": "./snippet/tyrano.snippet.json"
  }
]
}
  \end{lstlisting}
%%%%%%%%%%%%% 参考文献 %%%%%%%%%%%%%%%%%%
\begin{thebibliography}{9}
  \bibitem{syntaxes&snippet} 
  %bibitemの後ろにプレーンテキストを書かない
  vscodeで自作のシンタックスハイライト・スニペット拡張機能を作る\\
  https://qiita.com/OrukRed/items/03f0e38e0f8553ee35d2

  \bibitem{aboutsnippet}
  VSCodeでスニペットを作成する\\
  https://qiita.com/ygsiro/items/b8b9df4e55b51b844f82

\end{thebibliography}
\end{document}